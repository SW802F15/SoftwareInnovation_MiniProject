\section{VBM Review}
A VBM Review, short for Vision \& Business Model Review, should be done at the end of a sprint (or iteration), and is about evaluating the project in terms of four main dimensions:

\begin{itemize}
	\item What we see as the problem.
	\item What we see as the solution.
	\item Why we want to solve the problem.
	\item How we know we are doing it right.
\end{itemize}

These dimensions are closely related to the four Views: paradigm, product, project, and process \citep[pp. 44-47]{essence}.

Ideally each of the questions should be answered, but here we will focus on the Affordance Inquiry \citep[pp. 59-60]{essence}, which seeks to answer the second question. This affordance inquiry is based on the state of the application at the end of the last iteration.

\subsection*{Affordance Inquiry}
To keep the affordance inquiry simple, we will answer the generic questions stated in the book.\\

\noindent \textbf{Are there free sensors that may offer additional insights about status in the use context?}
Two sensors come to mind: GPS and heart rate monitor; the latter requiring either a smart watch or an external sensor.
Using the GPS several options for new features appear.
In terms of statistics it would allow for creating an overview of the user's speed, route, distance, and more.
In terms of creating work-out sessions, it would make it possible to use the application to help the user try and reach a certain goal by increasing or decreasing the tempo of the music, indicating to the user how fast they should be running to reach the goal.
Using a heart rate monitor it would be possible to create a workout session based on the user's pulse, thereby making sure the workout matches the user's fitness.
Additionally it could open up for creating a safety monitor, checking that the user has not suffered from a stroke.
This would of course require a way to verify the watch/sensor has not simply been taken off.\\

\noindent \textbf{Is all data available used in a relevant way?}\\
The data available consists of the user's pace (steps per minute), calculated from accelerometer readings, and of song info, including the song's tempo (beats per minute).
The user's pace is used for finding songs to put in the queue to play next, and in addition it is displayed on the screen as the user's current number of steps per minute.
Finding new songs to match the pace solves one of the main problems we are focusing on, but displaying the number of steps per minute on the screen does not necessarily provide much value for the user, other than indicating if it is indeed working.
Having too much data on the screen can seem confusing, but besides that it is not a disadvantage to be able to read the value.
The data could easily be used for a few more things, and two immediately come to mind.
The user may be interested in some more persistent statistics, such as a total number of steps within a given time period or -- with some assistance from the GPS to find the distance travelled -- their average step length, maybe even differentiated between walking, jogging, and running step lengths.
The statistics would be especially interesting if we were looking to take the application towards a focusing more on running rather than playing music.
More interestingly, the current pace of the user could help determine when music should start playing and when it should stop again.
If the goal is to match the music to the user's pace there is no sense in playing music before that pace has been determined.
On the other hand, it should be possible to play music that matches any pace, and even play music without moving.

Concerning the presentation of the data, it is currently only available to view on the screen, and most likely the user will not be able to look at the screen while running, and an alternative representation of the most important data could be advantageous.
One option for presenting it could be to use text-to-speech software to read it out loud.\\

\noindent \textbf{Is there any untapped potential in the design?}\\
The design focuses on solving the most important problem, namely that of creating a better running experience for the user by playing music with a tempo that fits the user's pace.
If we disregard the risk of bloating the application, making it more confusing to use, there are a number of potential features that could be implemented with the current design.

First of all it would be possible to
%Music->run vs Run->music
%Step counter
%Music player
%Calories / running info
%Upload song info










%At least one of the following VBM reviews (see Essence-book Chapter 6):
%\begin{itemize}
%\item Reflection Inquiry (see Essence-book Section 8.5).
%\item Affordance Inquiry (see Essence-book Section 9.5).
%\item Vision Inquiry (see Essence-book Section 10.5).
%\item Facilitation Inquiry (see Essence-book Section 11.5).
%\end{itemize}