\section{Idea Maturation}
%todo Give an example of idea maturation.
The idea behind idea maturation is that a problem is not static, and neither is the solution to it. Therefore, as detailed by Aaen  \citep[ pp. 159-162]{essence} the creative process should be iterative - the idea should develop over time.
As a means of maturing our idea, we have updated our configuration table to reflect our idea for the third release of the application.
The updated configuration table is shown on \Cref{table:config2} and the changes are explained here.

\begin{table}
\begin{tabular}{l|l|l|l|l}
         & \textbf{Paradigm} & \textbf{Product} & \textbf{Project} & \textbf{Process} \\ \hline
\parbox[t][4cm][c]{0.02\textwidth}{\rotatebox{90}{Focus}} %TODO find vertical height
	& \parbox[t]{0.20\textwidth}{\small 
		\textit{Reflection} \newline
		Challenge: \newline
		. \newline
		Use context: \newline
		Running while listening to music from a smartphone.
	}
	& \parbox[t]{0.20\textwidth}{\small 
		\textit{Affordance} \newline
		Running Pacer. \newline
		Option: \newline
		Interval Trainer / Training programs. \newline
		Step Counter (no music). \newline
		Music Player (no running). \newline
	}
	& \parbox[t]{0.20\textwidth}{\small  
		\textit{Vision} \newline
		Vision: \newline
		Running Pacer by use of music. \newline
		Use step counter to match song to running tempo.
	}
    & \parbox[t]{0.20\textwidth}{\small 
	    \textit{Facilitation} \newline
	    
   } \\ \hline
\parbox[t][5cm][c]{0.02\textwidth}{\rotatebox{90}{Overview}}
	& \parbox[t]{0.20\textwidth}{\small 
    \textit{Stakeholders} \newline
    Project group.
	}
	& \parbox[t]{0.20\textwidth}{\small 
		\textit{Design} \newline
	 	Running Pacer with music player-like design and functionalities.  
	}
	& \parbox[t]{0.20\textwidth}{\tiny
		\textit{\small  Elements} \newline
		Grounds: \newline
		Paced music gives better running experience. \newline
		Warrant: \newline
		When running, it is human nature to match pace with the music playing. \newline
		Qualifier: \newline
		Detailed music information (bpm) required. Precise SPM measurement required. \newline
		Rebuttal: \newline
		Use context: \newline
		To match songs with pace. \newline
		Limitations: \newline
		Step counter measures step frequency.
	}
	& \parbox[t]{0.20\textwidth}{\small 
		\textit{Evaluation} \newline
		Procedure: \newline Iteration retrospective by surrogate customer \newline
		Criteria: \newline Evaluate immediate functionality based on acceptance tests.
	}\\ \hline
\parbox[t][3.5cm][c]{0.02\textwidth}{\rotatebox{90}{Details}}
	& \parbox[t]{0.20\textwidth}{\small 
		\textit{Scenarios}\newline
		Automatically fades into songs, which fit running pace.\newline
		Use private collection of MP3 files as a basis for exercise/running.
	}
	& \parbox[t]{0.20\textwidth}{\small 
		\textit{Components}\newline
		Music player. \newline
		Music library. \newline
		Step Counter.
		
	}
	& \parbox[t]{0.20\textwidth}{\small 
		\textit{Features}\newline
		Running pacer.\newline
		Music player.\newline
		Step counting.
	}
	& \parbox[t]{0.20\textwidth}{\small 
		\textit{Findings}
	}\\ \hline     
\end{tabular}
\end{table}
%todo \begin{itemize}
%todo \item You might for example present a later Configuration Table from your project including a description of how the new table differ from the first, and why (see Essence-book Chapter 23).
%\end{itemize}

\begin{itemize}
\item \textbf{Paradigm Details}: Based on the findings from the 2nd release, a new scenario emerged: When the user is running with the screen off, it should still be possible to control the application.

\item \textbf{Product Overview}: The product design should include a way to interact with the application without a graphical interface.

\item \textbf{Project Details}: While the components are the same, a new feature is needed for the music player: A non-graphical user interface, allowing the user to navigate the application while running and without looking at the screen.

\item \textbf{Process Details}: Without any  feedback, e.g. a vibration, it is hard to determine if the command given in the non-graphical user interface was registered, making it harder to control properly. \newline
Additionally, it is necessary to take an appropriate action if the user stops running. This could be recording their last stable pace, and keep playing songs matching that tempo.
\end{itemize}