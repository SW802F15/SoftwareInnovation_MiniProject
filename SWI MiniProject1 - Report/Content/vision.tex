\section{Project Vision}
%todo
A representation of your Project Vision and (if relevant) Prospect Research (see Essence-book Chapter 15).
Explain your choice of representation type (see Essence-book Chapter 24).

%todo Characterization of your project using Vision Scenarios  (see Essence-book Part 4):
	%todo Describe your axes.
For our vision scenarios \cite[ p. 127]{essence} we have selected the following dimensions:
\begin{itemize}
\item Online vs Offline %Client Server vs Standalone
\item Narrow vs Wide
\end{itemize}

The \texttt{Online vs Offline} dimension points to variations in reliance on online resources. An online focus could shift  towards a server with a thin client application. Music streaming comes to mind and it would be possible to have a larger music libary available on the smartphone. A drawback of a shift towards online is a the it requires a stable internet connecting. An offline standalone application would rely on a local music libary and not be dependant on a stable internet connection in order to function. It would however be more difficult to acquire not commonly used information from local music files just a beat pr minute.


The \texttt{Narrow vs Wide} dimension points to variation in scope of the application. A narrow scope will result in a focused application covering just running. This will allow for more benefits specifiability suited for runners. One the other hand a wider scope will cover more general forms of exercise. For a user performing multi forms of exercise there might a benefit to having everything linked in the same application even if it is at the cost of more specialised features. A drawback of the multi purpose application is that it might be harder to operate.
%usabilly

From these dimensions we can identify four vision scenarios:

\begin{itemize}
\item %Online - Narrow
\item %Online - Wide
\item %Offline - Narrow
\item %Offline - Wide
\end{itemize}


% Set tempo from pace vs Set pace from tempo.




	%todo Characterize the project using at least one of the Essence roles.

