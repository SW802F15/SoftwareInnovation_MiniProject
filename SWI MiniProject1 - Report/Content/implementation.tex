\section{Implementation of Essence}
%todo A description of use context and selected use scenarios (see Essence-book Chapter 13).


%todo A discussion of how to implement support for key use scenarios (see Essence-book Chapter 14).

\subsection{Use context}
According to \cite[ p. 83]{essence} the definition of an use context is:
\begin{center}
	“\textit{Use context} - the physical, legal, and systemic world, where the scenario unfolds.
	 The locations, physical constraints, rules, regulations, preexisting systems, and technological platforms relevant to the design process.”
\end{center}

The use context for our project is a world where smartphones are used while running.
In this world people listen to music while running, through their smartphones.
Smartphones have built in accelerometer, GPS, internet access, etc..
Runners have their smartphones attached to their arms, limiting the view of the smartphones' screen.
The runners use headphones specialised for running, ensuring they can still hear the traffic while listening to music.

Running can take place both outdoor and indoor on a treadmill.
Indoor running limits some technological sensors, e.g. GPS, but enhances other features such as internet access through WiFi.
Outdoor running can in some places restrict the internet access, but enables the use of other sensors such as GPS.

Supplementary sensors can be accessed through other technological devices such as wearables.
Especially smartwatches are a popular and useful addition to the smartphone.

A lot of services are available through the internet.
Some of these services include beats-per-minute lookup for songs, streaming of music, and other information related to music.
