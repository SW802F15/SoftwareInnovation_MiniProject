\section{Evaluation}

%Theoretical evaluation. Compare your work with some of the theories that are referred to in the course literature. The evaluation should not be value judgment-based, but should seek to relate your work to the theories and principles that the course draws from. You might like to focus on 

%Your process - the steps, tools and practices that you used to arrive at your results (SI Chapter 4,7)

%creative requirements analysis
the designed process framework <- p68 - ESSENCE
%low tech prototyping
%user-driven software innovation
%community development
%the research prototype

support for escaping routine work <- p118 - All the tools
%sandbox tools
%knowledge tools
collaboration tools: internal (project), external (customers and other stakeholders) <- p118 - internal project (collaborative writing)
%visualization and overview support
%creativity technique support



%Your eventual vision. Did you experience major changes or discontinuities in the vision during your work? (SI Chapter 1,3)

%Your personal creative contribution (SI Chapter 5)

the developer’s mental process: recognising and exploiting discovery points <- p84 - breaks and sleep, burnout. p86 - adapting methodology

%a set of personal development competences concerned with both solving problems and recognising opportunities
%a style of thinking associated with different strengths in individual’s development personalities
%meta-thinking: recognising,  predispositions  and tendencies in one’s own (and others’) thinking and coming beyond them
%whole-brain thinking: beyond rationality
%a state of mind: the way the developer’s mind is disposed when being creative (flow)
%a relationship between the individual developer and communities of people and ideas (domain, field)
%a universal mental skill to be enhanced


%Team process and dynamics (SI Chapter 6) or some combination of these.
p100 - surrogate customer
p104 - focus  on  interaction, standup
