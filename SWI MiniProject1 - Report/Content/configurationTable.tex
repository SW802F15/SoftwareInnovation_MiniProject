\section{Configuration table (Early)}
%An early Configuration Table as outlined in the Essence book Part 1.
For our early configuration table we have our second release. The first release was very small and produced a music player which now needs to be expanded to become a running pacer, takings steps per minute into account when selecting the next song from a queue.
\begin{table}
\scriptsize
\begin{tabular}{|l|l|l|l|l|} \hline
         & \textbf{Paradigm} & \textbf{Product} & \textbf{Project} & \textbf{Process} \\ \hline
\parbox[t][4cm][c]{0.02\textwidth}{\rotatebox{90}{Focus}} %TODO find vertical height
	& \parbox[t][][c]{0.20\textwidth}{ 
		\textit{Reflection}
		\begin{itemize}[leftmargin=*]
		\item Challenge: Can we improve the running experience?
		\item Use context: Running while listening to music from a smartphone.
		\end{itemize}
	}
	& \parbox[t][][c]{0.20\textwidth}{ 
		\textit{Affordance} 
		\begin{itemize}[leftmargin=*]
		\item Running pacer. 
		\item Option: 
			Interval Trainer / Training programs. \newline
			Step Counter (no music). \newline
			Music Player (no running). 
		\end{itemize}	 
	}
	& \parbox[t][][c]{0.20\textwidth}{  
		\textit{Vision} 
		\begin{itemize}[leftmargin=*]
		\item Vision: Running pacer by use of music. 
		\item Use step counter to match song to running pace.
		\end{itemize}
	}
    & \parbox[t][][c]{0.20\textwidth}{ 
	    \textit{Facilitation}
   		\begin{itemize}[leftmargin=*]
		\item Focus on immediate benefits to user.
		\end{itemize}
   } \\ \hline
\parbox[t][5cm][c]{0.02\textwidth}{\rotatebox{90}{Overview}}
	& \parbox[t][][c]{0.20\textwidth}{ 
	    \textit{Stakeholders} 
		\begin{itemize}[leftmargin=*]
		\item A runner who enjoys music.
		\end{itemize}
	}
	& \parbox[t][][c]{0.20\textwidth}{ 
		\textit{Design} 
		\begin{itemize}[leftmargin=*]
		\item Running pacer with music player-like design and functionalities.  
		\end{itemize}
	}
	& \parbox[t][][c]{0.20\textwidth}{
		\textit{Elements}
		\begin{itemize}[leftmargin=*] 
		\item Grounds: 
		Music that fits your pace provides a better running experience. 
		\item Warrant: 
		When running it is human nature to match pace with the tempo of the music playing. 
		\item Qualifier:
		Songs matching current pace required at all times.
		\item Rebuttal: \newline
		Use context: \newline
		Changing running pace during a song. \newline 
		Limitations: 
		It is necessary to alter the music's tempo or transition to other songs.
		\end{itemize}
	}
	& \parbox[t][][c]{0.20\textwidth}{ 
		\textit{Evaluation}
		\begin{itemize}[leftmargin=*]
		\item Procedure:  Iteration review with surrogate customer.
		\item Criteria:  Evaluate immediate functionality based on acceptance tests.
		\end{itemize}		
	}\\ \hline
\parbox[t][3.5cm][c]{0.02\textwidth}{\rotatebox{90}{Details}}
	& \parbox[t][][c]{0.20\textwidth}{ 
		\textit{Scenarios}
		\begin{itemize}[leftmargin=*]
		\item Automatically fade into songs, which fit running pace.
		\item Use personal collection of music files as a basis for exercise/running.
		\end{itemize}
	}
	& \parbox[t][][c]{0.20\textwidth}{ 
		\textit{Components}
		\begin{itemize}[leftmargin=*]
		\item Music player
		\item Music library
		\item Step counter
		\end{itemize}
	}
	& \parbox[t][][c]{0.20\textwidth}{ 
		\textit{Features}
		\begin{itemize}[leftmargin=*]
		\item Running pacer
		\item Music player
		\item Step counter
		\end{itemize}
	}
	& \parbox[t][][c]{0.20\textwidth}{ 
		\textit{Findings} 
		\begin{itemize}[leftmargin=*]
		\item Controlling the device while running can be difficult.
		\end{itemize}
	}\\ \hline     
\end{tabular}
\caption[Table caption text]{Configuration table for 2nd release.}
\label{table:config1}
\end{table}

% A systematic description of the Configuration Table explaining the overall idea and how it relates to your problem.


\begin{itemize}
\item \textbf{Paradigm Focus}: The challenge is to improve the running experience by matching the tempo of the music with the pace of the runner. The application is supposed to be used while running.

\item \textbf{Paradigm Overview}: Stakeholder is the user who is using the application, i.e., the runner.

\item \textbf{Paradigm Details}: For this release there are two main scenarios to fulfil. Firstly the application should automatically select appropriate songs (songs with a matching BPM) to play as the user runs at varying paces. Secondly it should be possible to customise the song pool from which songs are selected.

\item \textbf{Product Focus}: The application is first and foremost a running pacer, meaning the purpose is to match music tempo with the user's pace, measured in steps per minute. It is possible to add an option for interval training so the application no longer adapt the music's tempo based on pace, but encourage the user to adapt his or her pace to predetermined intervals of low and high tempo music. Additional training programs could also be added. The application can also serve as a step counter, not playing any music but just recording steps taken and displaying it as either total steps or steps per minute. Similarly it can just play music without taking SPM into account.

\item \textbf{Product Overview}: The application design should be based on a familiar music player design, and it should have the expected functionally from a music player, but new elements (such as SPM) should expand on the familiar design. The navigation buttons should be placed, so they are easy to use while moving.

\item \textbf{Product Details}: This configuration consists of 3 components: the music player, a music library, and a step counter. The music player is used to play songs and serve as the primary interface for the user. The music library is one or more folders with music files on the device. The step counter makes use of the device's accelerometer to register steps taken.

\item \textbf{Project Focus}: The overall vision of the application is as a running pacer which matches music to the user's running pace.

\item \textbf{Project Overview}: \newline
\textit{Grounds:} Music can have an influence on the running experience. Having music that matches your running pace makes sure it does not have a negative impact on the exercise session, and having the application find appropriate music automatically makes the process of finding good running music less tedious. \newline
\textit{Warrant}: Many people prefer running while listening to music, and it will often result in people trying to match their running pace with the music playing. \newline
\textit{Qualifier}: It is important that the song playing always, or as much as possible, matches the user's pace, so their running experience is not disrupted. \newline
\textit{Rebuttal}: The user may change pace in the middle of a song, and this poses a problem. It could be solved by altering the music's tempo just enough to match the new pace, without distorting the sound of the music too much. Another solution could be to transition to another piece of music, when the new pace is confirmed stable.

\item \textbf{Project Details}: The application should work as a running pacer, a music player, and a step counter.

\item \textbf{Process Focus}: Through an iterative process there will be focus on the most important features as decided by the customer, providing immediate benefits to the user.

\item \textbf{Process Overview}: We evaluate the application from the customer's point of view at the end of each iteration. The evaluation is based on the fulfilment of acceptance tests.

\item \textbf{Process Details}: While running it can be difficult to navigate the application, as there is limited feedback when using a touch screen. Instead it should be possible to use gestures or taps to control the application, giving it a more 
consistent user experience.
\end{itemize}