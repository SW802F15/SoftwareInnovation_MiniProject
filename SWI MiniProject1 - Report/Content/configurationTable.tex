\section{Configuration table (Early)}
%An early Configuration Table as outlined in the Essence book Part 1.
For our early configuration table we have our second release. The first release produced a music player which now needs to be expanded to become a running pacer, takings steps per minute into account when selecting the next song from a queue.
\begin{table}
\scriptsize
\begin{tabular}{|l|l|l|l|l|} \hline
         & \textbf{Paradigm} & \textbf{Product} & \textbf{Project} & \textbf{Process} \\ \hline
\parbox[t][4cm][c]{0.02\textwidth}{\rotatebox{90}{Focus}} %TODO find vertical height
	& \parbox[t][][c]{0.20\textwidth}{ 
		\textit{Reflection}
		\begin{itemize}[leftmargin=*]
		\item Challenge: Can we improve the running experience?
		\item Use context: Running while listening to music from a smartphone.
		\end{itemize}
	}
	& \parbox[t][][c]{0.20\textwidth}{ 
		\textit{Affordance} 
		\begin{itemize}[leftmargin=*]
		\item Running pacer. 
		\item Option: 
			Interval Trainer / Training programs. \newline
			Step Counter (no music). \newline
			Music Player (no running). 
		\end{itemize}	 
	}
	& \parbox[t][][c]{0.20\textwidth}{  
		\textit{Vision} 
		\begin{itemize}[leftmargin=*]
		\item Vision: Running pacer by use of music. 
		\item Use step counter to match song to running pace.
		\end{itemize}
	}
    & \parbox[t][][c]{0.20\textwidth}{ 
	    \textit{Facilitation}
   		\begin{itemize}[leftmargin=*]
		\item Focus on immediate benefits to user.
		\end{itemize}
   } \\ \hline
\parbox[t][5cm][c]{0.02\textwidth}{\rotatebox{90}{Overview}}
	& \parbox[t][][c]{0.20\textwidth}{ 
	    \textit{Stakeholders} 
		\begin{itemize}[leftmargin=*]
		\item A runner who enjoys music.
		\end{itemize}
	}
	& \parbox[t][][c]{0.20\textwidth}{ 
		\textit{Design} 
		\begin{itemize}[leftmargin=*]
		\item Running pacer with music player-like design and functionalities.  
		\end{itemize}
	}
	& \parbox[t][][c]{0.20\textwidth}{
		\textit{Elements}
		\begin{itemize}[leftmargin=*] 
		\item Grounds: 
		Music that fits your pace provides a better running experience. 
		\item Warrant: 
		When running it is human nature to match pace with the tempo of the music playing. 
		\item Qualifier:
		Songs matching current pace required at all times.
		\item Rebuttal: \newline
		Use context: \newline
		Changing running pace during a song. \newline 
		Limitations: 
		It is necessary to alter the music's tempo or transition to other songs.
		\end{itemize}
	}
	& \parbox[t][][c]{0.20\textwidth}{ 
		\textit{Evaluation}
		\begin{itemize}[leftmargin=*]
		\item Procedure:  Iteration review with surrogate customer.
		\item Criteria:  Evaluate immediate functionality based on acceptance tests.
		\end{itemize}		
	}\\ \hline
\parbox[t][3.5cm][c]{0.02\textwidth}{\rotatebox{90}{Details}}
	& \parbox[t][][c]{0.20\textwidth}{ 
		\textit{Scenarios}
		\begin{itemize}[leftmargin=*]
		\item Automatically fade into songs, which fit running pace.
		\item Use personal collection of music files as a basis for exercise/running.
		\end{itemize}
	}
	& \parbox[t][][c]{0.20\textwidth}{ 
		\textit{Components}
		\begin{itemize}[leftmargin=*]
		\item Music player
		\item Music library
		\item Step counter
		\end{itemize}
	}
	& \parbox[t][][c]{0.20\textwidth}{ 
		\textit{Features}
		\begin{itemize}[leftmargin=*]
		\item Running pacer
		\item Music player
		\item Step counter
		\end{itemize}
	}
	& \parbox[t][][c]{0.20\textwidth}{ 
		\textit{Findings} 
		\begin{itemize}[leftmargin=*]
		\item Controlling the device while running can be difficult.
		\end{itemize}
	}\\ \hline     
\end{tabular}
\caption[Table caption text]{Configuration table for 2nd release.}
\label{table:config1}
\end{table}

%todo A systematic description of the Configuration Table explaining the overall idea and how it relates to your problem.


\begin{itemize}
\item Paradigm Focus: The challenge is as stated in the intro to improve running experience by matching music beat to pace. The application is supposed to be used while running.
\item Paradigm Overview: Unfortunately the user who came up with the idea for the project was unable to act as a customer for the project so the only stakeholder is the group itself.
\item Paradigm Details: For this release we have a two main scenarios to fulfil. First we want the application to automatically select appropriate (songs with a matching beat) songs to play as the user runs at varying paces. The song pool to select from should be customisable for each device.
\item Product Focus: The application is first and foremost a running pace, meaning the purpose is to match music beat with steps per minute (pace). It is possible to add an option for interval training so the application no longer adapt music beat based on pace, but encourage the user to adapts his or her pace to predetermined intervals of low and high beat music. Additional training programs could also be added. Further more the application can also server as a step counter not playing any music just recording steps taken and displays steps per minute, and the other way around just playing music without taking steps per second into account.
\item Product Overview: The application design should be based on a ``familiar music player design'', it should have the expected functionally from a music player and new elements (such as steps per minute) should expand on the familiar design. 
%todo consider ``familiar music player design''
\item Product Details: This configuration consists of 3 components the music player, a library and a step counter. The music player is used to play songs and serve as the primary interface for the user. The music library is one or more folders with MP3 files on the device. The step counter makes use of the device's accelerometer to register steps taken.
\item Project Focus: The overall vision of the application is as a running pacer which matches the beat of songs to played with running pace.
\item Project Overview: 
\item Project Details: 
\item Process Focus:
\item Process Overview:
\item Process Details:

\end{itemize}
